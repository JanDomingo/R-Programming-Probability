\documentclass{article}\usepackage[]{graphicx}\usepackage[]{color}
%% maxwidth is the original width if it is less than linewidth
%% otherwise use linewidth (to make sure the graphics do not exceed the margin)
\makeatletter
\def\maxwidth{ %
  \ifdim\Gin@nat@width>\linewidth
    \linewidth
  \else
    \Gin@nat@width
  \fi
}
\makeatother

\definecolor{fgcolor}{rgb}{0.345, 0.345, 0.345}
\newcommand{\hlnum}[1]{\textcolor[rgb]{0.686,0.059,0.569}{#1}}%
\newcommand{\hlstr}[1]{\textcolor[rgb]{0.192,0.494,0.8}{#1}}%
\newcommand{\hlcom}[1]{\textcolor[rgb]{0.678,0.584,0.686}{\textit{#1}}}%
\newcommand{\hlopt}[1]{\textcolor[rgb]{0,0,0}{#1}}%
\newcommand{\hlstd}[1]{\textcolor[rgb]{0.345,0.345,0.345}{#1}}%
\newcommand{\hlkwa}[1]{\textcolor[rgb]{0.161,0.373,0.58}{\textbf{#1}}}%
\newcommand{\hlkwb}[1]{\textcolor[rgb]{0.69,0.353,0.396}{#1}}%
\newcommand{\hlkwc}[1]{\textcolor[rgb]{0.333,0.667,0.333}{#1}}%
\newcommand{\hlkwd}[1]{\textcolor[rgb]{0.737,0.353,0.396}{\textbf{#1}}}%
\let\hlipl\hlkwb

\usepackage{framed}
\makeatletter
\newenvironment{kframe}{%
 \def\at@end@of@kframe{}%
 \ifinner\ifhmode%
  \def\at@end@of@kframe{\end{minipage}}%
  \begin{minipage}{\columnwidth}%
 \fi\fi%
 \def\FrameCommand##1{\hskip\@totalleftmargin \hskip-\fboxsep
 \colorbox{shadecolor}{##1}\hskip-\fboxsep
     % There is no \\@totalrightmargin, so:
     \hskip-\linewidth \hskip-\@totalleftmargin \hskip\columnwidth}%
 \MakeFramed {\advance\hsize-\width
   \@totalleftmargin\z@ \linewidth\hsize
   \@setminipage}}%
 {\par\unskip\endMakeFramed%
 \at@end@of@kframe}
\makeatother

\definecolor{shadecolor}{rgb}{.97, .97, .97}
\definecolor{messagecolor}{rgb}{0, 0, 0}
\definecolor{warningcolor}{rgb}{1, 0, 1}
\definecolor{errorcolor}{rgb}{1, 0, 0}
\newenvironment{knitrout}{}{} % an empty environment to be redefined in TeX

\usepackage{alltt}
\usepackage[sc]{mathpazo}
\renewcommand{\sfdefault}{lmss}
\renewcommand{\ttdefault}{lmtt}
\usepackage[T1]{fontenc}
\usepackage{geometry}
\geometry{verbose,tmargin=2.5cm,bmargin=2.5cm,lmargin=2.5cm,rmargin=2.5cm}
\setcounter{secnumdepth}{2}
\setcounter{tocdepth}{2}
\usepackage[unicode=true,pdfusetitle,
 bookmarks=true,bookmarksnumbered=true,bookmarksopen=true,bookmarksopenlevel=2,
 breaklinks=false,pdfborder={0 0 1},backref=false,colorlinks=false]
 {hyperref}
\hypersetup{
 pdfstartview={XYZ null null 1}}

\makeatletter
%%%%%%%%%%%%%%%%%%%%%%%%%%%%%% User specified LaTeX commands.
\renewcommand{\textfraction}{0.05}
\renewcommand{\topfraction}{0.8}
\renewcommand{\bottomfraction}{0.8}
\renewcommand{\floatpagefraction}{0.75}

\makeatother
\IfFileExists{upquote.sty}{\usepackage{upquote}}{}
\begin{document}








The results below are generated from an R script.

\begin{knitrout}
\definecolor{shadecolor}{rgb}{0.969, 0.969, 0.969}\color{fgcolor}\begin{kframe}
\begin{alltt}
\hlcom{#Jan Domingo}
\hlcom{#Homework #8 Appendix: R Programs}
\hlcom{#4, 6, and 7}

\hlcom{#4  Babies' birth weights are normally distributed with mean 120 ounces and standard deviation 20 ounces. }
\hlcom{#   Low birth weight is an important indicator of a newborn baby's chances of survivlal.}
\hlcom{#   One definition of low birth weight is that it is the fifth percentile of the weight distribution.}

\hlcom{#(a): Using R find the value of the fifth percentile in which babies who weigh less than this amount would be }
\hlcom{#     considered low birth weight. p = .05, mean = 120oz, sd = 20}
\hlkwd{qnorm}\hlstd{(}\hlnum{0.05}\hlstd{,} \hlnum{120}\hlstd{,} \hlnum{20}\hlstd{,} \hlkwc{lower.tail} \hlstd{=} \hlnum{TRUE}\hlstd{)} \hlcom{# = 87.10293 ounces}
\end{alltt}
\begin{verbatim}
## [1] 87.10293
\end{verbatim}
\begin{alltt}
\hlcom{#(b): Very low birth weight is used to describe babies who are born weighing less than 52 ounces. Using R,}
\hlcom{#     find the probability that a baby is born with very low birth weight.}
\hlkwd{pnorm}\hlstd{(}\hlnum{52}\hlstd{,} \hlnum{120}\hlstd{,} \hlnum{20}\hlstd{,} \hlkwc{lower.tail} \hlstd{=} \hlnum{TRUE}\hlstd{)} \hlcom{# = .0034}
\end{alltt}
\begin{verbatim}
## [1] 0.0003369293
\end{verbatim}
\begin{alltt}
\hlcom{#************************************************************************************************************}


\hlcom{#6  The CPU time T, in seconds, to execute a piece of software changes based on the input parameters. }
\hlcom{#   Suppose the CPU time follows a Weibull distribution with parameters alpha = 0.05 and beta = 0.25.}
\hlcom{#   Find f(t), E(t), and Var(t) (Use the gamma in R for the latter two. Also recall alpha deteermines}
\hlcom{#   the scale, beta determines the shape). Also find the probability that the CPU software will take}
\hlcom{#   longer than 1 second to execture}
\hlcom{#   alpha = 0.05, beta = 0.25}

\hlcom{#E(t): alpha * gamma(1/beta + 1)}
\hlnum{0.05} \hlopt{*} \hlkwd{gamma}\hlstd{(}\hlnum{1}\hlopt{/}\hlnum{0.25} \hlopt{+} \hlnum{1}\hlstd{)} \hlcom{# = 1.2}
\end{alltt}
\begin{verbatim}
## [1] 1.2
\end{verbatim}
\begin{alltt}
\hlcom{#Var(t): alpha^2 * gamma(2/beta + 1) - alpha^2 * gamma(1/beta + 1)^2}
\hlnum{0.05}\hlopt{^}\hlnum{2} \hlopt{*} \hlkwd{gamma}\hlstd{(}\hlnum{2}\hlopt{/}\hlnum{0.25} \hlopt{+} \hlnum{1}\hlstd{)} \hlopt{-} \hlnum{0.05}\hlopt{^}\hlnum{2} \hlopt{*} \hlstd{(}\hlkwd{gamma}\hlstd{(}\hlnum{1}\hlopt{/}\hlnum{0.25} \hlopt{+} \hlnum{1}\hlstd{))}\hlopt{^}\hlnum{2} \hlcom{# = 99.36}
\end{alltt}
\begin{verbatim}
## [1] 99.36
\end{verbatim}
\begin{alltt}
\hlcom{#Probability the CPU softare will take longer than 1 second to execute}
\hlnum{1} \hlopt{-} \hlkwd{pweibull}\hlstd{(}\hlnum{1}\hlstd{,} \hlnum{0.25}\hlstd{,} \hlnum{0.05}\hlstd{)} \hlcom{# = 0.1206644}
\end{alltt}
\begin{verbatim}
## [1] 0.1206644
\end{verbatim}
\begin{alltt}
\hlcom{#************************************************************************************************************}

\hlcom{#7  A PE instructor studied the verical distances that students in a ninth grade class could jump which he found}
\hlcom{#   follows a Weibull distribution with parameters alpha = 11.08 and beta = 3.7. What is the mean (again use the}
\hlcom{#   gamma function to help find the mean)? Using R, what range of distances constitute the middle 50% of the }
\hlcom{#   distribution?}
\hlcom{#   alpha = 11.08, beta = 3.7}

\hlcom{# Computing the mean:(alpha/beta) * gamma(1/alpha)}
\hlstd{(}\hlnum{11.08}\hlopt{/}\hlnum{3.7}\hlstd{)} \hlopt{*} \hlkwd{gamma}\hlstd{(}\hlnum{1}\hlopt{/}\hlnum{11.08}\hlstd{)} \hlcom{# = 31.69874}
\end{alltt}
\begin{verbatim}
## [1] 31.69874
\end{verbatim}
\begin{alltt}
\hlcom{#Range of Distances for middle 50% of distribution:}

\hlcom{#At 75%}
\hlkwd{qweibull}\hlstd{(}\hlnum{0.75}\hlstd{,} \hlnum{11.08}\hlstd{,} \hlnum{3.7}\hlstd{,} \hlkwc{lower.tail} \hlstd{=} \hlnum{TRUE}\hlstd{)} \hlcom{# = 3.810698}
\end{alltt}
\begin{verbatim}
## [1] 3.810698
\end{verbatim}
\begin{alltt}
\hlcom{#At 25%}
\hlkwd{qweibull}\hlstd{(}\hlnum{0.25}\hlstd{,} \hlnum{11.08}\hlstd{,} \hlnum{3.7}\hlstd{,} \hlkwc{lower.tail} \hlstd{=} \hlnum{TRUE}\hlstd{)} \hlcom{# = 3.306489}
\end{alltt}
\begin{verbatim}
## [1] 3.306489
\end{verbatim}
\end{kframe}
\end{knitrout}

The R session information (including the OS info, R version and all
packages used):

\begin{knitrout}
\definecolor{shadecolor}{rgb}{0.969, 0.969, 0.969}\color{fgcolor}\begin{kframe}
\begin{alltt}
\hlkwd{sessionInfo}\hlstd{()}
\end{alltt}
\begin{verbatim}
## R version 3.5.2 (2018-12-20)
## Platform: x86_64-apple-darwin15.6.0 (64-bit)
## Running under: macOS Mojave 10.14.3
## 
## Matrix products: default
## BLAS: /Library/Frameworks/R.framework/Versions/3.5/Resources/lib/libRblas.0.dylib
## LAPACK: /Library/Frameworks/R.framework/Versions/3.5/Resources/lib/libRlapack.dylib
## 
## locale:
## [1] en_US.UTF-8/en_US.UTF-8/en_US.UTF-8/C/en_US.UTF-8/en_US.UTF-8
## 
## attached base packages:
## [1] stats     graphics  grDevices utils     datasets  methods   base     
## 
## other attached packages:
## [1] knitr_1.22
## 
## loaded via a namespace (and not attached):
## [1] compiler_3.5.2 magrittr_1.5   tools_3.5.2    stringi_1.4.3  highr_0.7     
## [6] stringr_1.4.0  xfun_0.5       evaluate_0.13
\end{verbatim}
\begin{alltt}
\hlkwd{Sys.time}\hlstd{()}
\end{alltt}
\begin{verbatim}
## [1] "2019-05-09 12:57:08 PDT"
\end{verbatim}
\end{kframe}
\end{knitrout}


\end{document}
